% Created 2022-11-27 Sun 18:07
% Intended LaTeX compiler: pdflatex
\documentclass[11pt]{article}
\usepackage[utf8]{inputenc}
\usepackage[T1]{fontenc}
\usepackage{graphicx}
\usepackage{longtable}
\usepackage{wrapfig}
\usepackage{rotating}
\usepackage[normalem]{ulem}
\usepackage{amsmath}
\usepackage{amssymb}
\usepackage{capt-of}
\usepackage{hyperref}
\author{Malte Schaffeld}
\date{\today}
\title{Klausur 22 12}
\hypersetup{
 pdfauthor={Malte Schaffeld},
 pdftitle={Klausur 22 12},
 pdfkeywords={},
 pdfsubject={},
 pdfcreator={Emacs 28.2 (Org mode 9.6)}, 
 pdflang={English}}
\begin{document}

\maketitle
\tableofcontents


\section{12 Kommunikationssperren von Thomas Gordon}
\label{sec:org44aae26}
\subsection{Die Sperren}
\label{sec:org4e5e457}
\begin{enumerate}
\item befehlen, anleiten und kommandieren
``Mach deine Hausaufgaben. Hol mir ein Bier.''
\item warnen, drohen, ermahnen
``Wenn du das machst gibt es Probleme.''
\item zureden, moralisieren, predigen
``Sie dürfen nicht nur an sich selbst denken. In dieser Firma halten wir zusammen.''
\item beraten, Vorschläge machen, Lösungen geben
``mir geht es nicht gut.'' -``Geh zum Arzt.''
\item Vorhaltungen machen, belehren
``ich habe so viel zu tun.'' - ``Ach so schlimm ist das doch überhaupt nicht. Ich habe damals auch alles Hinbekommen.''
\item verurteilen, kritisieren, widersprechen, beschuldigen
``mir geht es nicht gut.'' -``Du verdirbst uns den ganzen Abend.''
\item loben, zustimmen
``ich habe keine Zeit dafür.'' - ``Aber sie sind mein Bester Mann für den Job''
\item beschimpfen, lächerlich machen, beschämen
``mir geht es nicht gut.'' - ``Stell dich nicht so an''
\item interpretieren, analysieren, diagnostizieren
``Anja hat keine Zeit mehr für mich'' - ``Du bist doch nur Neidisch weil sie einen Freund hat.''
\item beruhigen, bemitleiden, trösten
``Ich bin so traurig, mein Hund ist tot.'' - ``Na komm, er war doch schon so alt. Wie wäre es wenn wir morgen einen neuen Hund kaufen?''
\item forschen, fragen, verhören
``Es ist schwer Arbeit und Kinder unter einen Hut zu bringen.'' - ``Warum Arbeitest du überhaupt so schnell schon wieder?''
\item zurückziehen, ablenken, ausweichen
``Wir müssen reden.'' - `` Ich muss erstmal was Essen, wir können da ein anderes Mal drüber reden.
\end{enumerate}
\subsection{Bedeutung}
\label{sec:orgdb42064}
\begin{itemize}
\item diese Gesprächselemente bergen das Risiko sich negativ auf ein Gespräch auszuwirken, besonders wenn das Gespräch unter Stress stattfindet.
\item können den Gesprächspartner in Defensive treiben, oder zum schweigen veranlassen.
\end{itemize}
\section{Körperliche Signale deuten}
\label{sec:org3fa6d7c}

\section{5 Axiome von Paul Watzlawick}
\label{sec:org68ab5dc}
\begin{enumerate}
\item Man kann nicht nicht kommunizieren
sobald sich Menschen gegenseitig wahrnehmen, kommunizieren sie.
\item Jede Kommunikation hat einen Inhalts und einen Beziehungsaspekt
\begin{enumerate}
\item Inhalt:
was wird Inhaltlich mitgeteilt
\item Beziehung:
Wie soll die Nachricht verstanden werden? Ausdruck durch Gestik Mimik Tonfall und auch das Verhältnis der Gesprächspartner
\end{enumerate}

Kommunikation funktioniert nur dann erfolgreich wenn Inhalt und Beziehung getrennt gesehen und die Beziehungsebene so aufgenommen wird, wie sie vom Sender angedacht war.
\item Kommunikation ist Ursache und Wirkung
Kommunikationspartner reagieren fortlaufend auf einander, Kommunikation ist Kreisförmig
\item Kommunikation ist analog und digital
\begin{enumerate}
\item digital: verbal
sachlich objektive Wörter und Sätze, die sich auf bestimmte Sachverhalte beziehen. Vermitteln Informationen und lassen keinen Spielraum für Interpretationen
\item analog: nonverbal
gestik, mimik, beziehungen\ldots{} hiermit können zB Informationen zwischen den Zeilen vermittelt werdne.
\end{enumerate}

Im Idealfall sollten sich die beiden Ebenen nicht widersprechen.
\item Kommunikation ist Symmetrisch und Komplementär
\begin{enumerate}
\item symmetrisch
Gespräch auf Augenhöhe, Gesprächspartner wollen Gleichheit bestehen lassen, oder zu ihr zurückzukehren
\item komplementär
oft gibt es Über und Untergeordnete Gesprächsteilnehmer
\end{enumerate}
\end{enumerate}
\section{4 Schabel-Ohren-Modell}
\label{sec:orgc76df67}
Modell nach Friedmann Schulz von Thun.
\begin{itemize}
\item Vier Ebenen auf denen etwas verstanden werden kann.
\item Die Nachricht besteht aus den Vier Teilaspekten
\item Die nachricht kann von Sender und Empfänger unterschiedlich gemeint oder verstanden werden.
\end{itemize}


\begin{enumerate}
\item Sachebene: Sachinformationen
\item Beziehungsebene: Wie stehen die beiden Seiten zu einander?
\item Selbstkundgabe: Offen oder indirekt sagt der Sprecher etwas über sich aus.
\item Appellebene: Welche Wirkung wird erwünscht?
\end{enumerate}
\section{Eisbergmodell}
\label{sec:orgddf6a60}
\subsection{Zusammefassung}
\label{sec:org288f416}
Das Eisbergmodell beschreibt die Ansichten von Sigmund Freud zur Kommunikation.
Nur ein kleiner Teil der Kommunikation findet komplett bewusst statt. Dies ist im Modell der sichtbare Teil des Eisbergs (20\%).
Der deutlich größere Teil spielt sich im Modell unterhalb der Wasseroberfläche ab. Hier werden alle unbewussten Aspekte der Kommunikation Zusammengefasst (Beziehungsebene 80\%).

\begin{center}
\begin{tabular}{lll}
\hline
Sachebene & Sachinformationen & bewusste Kommunikation\\
 &  & verbale Botschaft\\
 &  & 20\%\\
\hline
Beziehungsebene & Werte, Gefühle, Wünsche & unbewusse\\
 & Einstellungen, frühere Erlebnisse & nonverbale Botschaft\\
 & Status, Ängste, Beurteilungen, & 80\%\\
 & Vorurteile & \\
\hline
\end{tabular}
\end{center}
\end{document}
